% vim:ft=tex

\section{Collaboration Technologies}

Since we want to ensure efficiency which includes that everyone has access to required data, can exchange data and that data can be versioned,  we decided to work with several technologies which will be explained further in the following sections.
\subsection{IceScrum}
As mentioned in the chapter before the development process should be done by using Scrum methods.
Those methods includes \emph{Sprint Plannings, Daily Scrums, Retrospectives} and \emph{Sprint Reviews}.
To apply the methods properly it was very helpful to use \emph{IceScrum} which is a web application in order to use Scrum. It provides storyboards on which tasks can be accepted. Furthermore it helps to keep track of progress and remaining tasks by burndowncharts as well as reports. We used IceScrum mainly for the sprint planning. To be precise for writing, estimating, prioritizing tasks, maintain our backlog and to distribute the incurred tasks as well as keeping track of our remaining story points and our progress.
\\IceScrum offers furthermore different roles which should be satisfied in Scrum. 
There is the \emph{Team} which is responsible to deliver the committed delivery in time and with the defined quality. In our case the Team consists of Anass Khaldi, Guillaume Goni and Pascal Riedel. Within the team there is one \emph{Scrum Master}, this role was taken by Kathi Rodi. The Scrum Master is responsible to ensure the team keeps to the values and practices of Scrum, sort of like a coach. He  removes impediments, facilitates meetings and work with product owners as well. This leads us to the next role in Scrum the \emph{Product Owner} which is the project`s key stakeholder. The Product Owner does not get to determine how much work happens in the sprint cycles, or alter the goals for that sprint. Product Owners must be available to the team, and engage actively with it. In this project the Product Owners are Prof. Dr. von Schwerin, Prof. Dr. Herbort and Prof. Dr. Goldstein.
\subsection{Box}
Besides IceScrum we also used \emph{Box} for collaborating.  Our Box includes for example the documentation of our daily scrums. We created a template which is used to log every daily meeting. According burndowncharts and reports will be stored there as well. Furthermore we used the box for recording our working hours and to exchange documents.
\subsection{GitHub}
To group our code base and have versioned on it, we created a GitHub account which is available at 
\href{https://github.com/dataBikeHsUlm}{https://github.com/dataBikeHsUlm}.
For the sake of clarity we created three repositories ordered by the main topics:
\begin{itemize}
\item WebApp: for the web server
\item NominatimLibrary: for the Jupyter notebooks
\item MapReduce: for the scripts for Hadoop
\end{itemize}
\subsection{Skype}
We carried out our daily scrums every monday at 4 pm as well as every thursday at 1.30 pm. On thursdays we have lectures together the whole day, hence the daily scrums on thursdays takes place at the university. Since it wasn't`t possible for the whole team to meet mondays as well, we decided to hold those daily scrums via Skype.
